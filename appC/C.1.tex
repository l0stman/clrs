\documentclass[a4paper,12pt]{article}
\usepackage{algorithmic}
\newcommand{\newpar}[1]
{\bigskip \noindent \textbf{Exercises #1} \newline}
\newcommand{\newprob}[1]
{\bigskip \noindent \textbf{Problem #1} \newline}
\newcommand{\subpar}[1]
{\medskip \noindent #1.}
\newcommand{\la}{\leftarrow}
\newcommand{\ra}{\rightarrow}

\begin{document}
\newpar{C.1-1}
The number of $k$-substrings in an $n$-string is $n-k+1$.  Thus the
total number of substrings of an $n$-string is
\[ \sum_{i=0}^n (n-i+1) = \sum_{i=1}^{n+1}i = \frac{(n+1)(n+2)}{2}.\]

\newpar{C.1-2}
There are $2^{2^n}$ $n$-input, $1$-output boolean functions.

There are $(2^m)^{2^n} = 2^{m2^n}$ $n$-input, $m$-output boolean
functions.

\newpar{C.1-3}
Let's consider one of the professor.  The place of this professor
doesn't matter because all the configurations obtained by rotation are
equivalent .  What matter is the place of the other $n-1$ relative to
him.  Those there's $(n-1)!$ ways to sit around a circular conference
table for the professors.

\newpar{C.1-4}
The sum of three distinct numbers is even if the first number is even
and the other two have the same parity or the first one is odd and the
other two have different parity.  So the number of ways to choose  those
numbers if their order matters is
\[ 50 \,49\, 48 + 50\, 50\, 49 + 50\, 49\, 50 + 50\, 50\, 49 = 485100.\]
But given that the order doesn't matter, we should divide this total
by the number of permutations of these three distinct numbers.
\[ \frac{485100}{6!} = 80850.\]

\newpar{C.1-5}
We have for $0 < k \le n$
\begin{eqnarray*}
  C_n^k &=& \frac{n!}{k!(n-k)!} \\
  &=& \frac{n}{k}\,\frac{(n-1)!}{(k-1)!(n-k)!} \\
  &=& \frac{n}{k}C_{n-1}^{k-1}
\end{eqnarray*}

\newpar{C.1-6}
We have for $0 \le k < n$
\begin{eqnarray*}
  C_n^k &=& \frac{n!}{k!(n-k)!} \\
  &=& \frac{n}{n-k}\,\frac{(n-1)!}{k!(n-k-1)!} \\
  &=& \frac{n}{n-k}C_k^{n-1}
\end{eqnarray*}

\newpar{C.1-7}
If the distinguished object isn't chosen, all the $k$ objects are
 chosen from the other $n-1$ objects.  And there's $C_{n-1}^k$ to
choose $k$ objects from $n$.

If the distinguished object is chosen,  the remaining $k-1$ objects
are chosen from the other $n-1$ objects remaining.  And there's
$C_{n-1}^{k-1}$ to do so.

So finally the number of ways to choose $k$ objects from $n$ is the
sum of the two previous case, thus
\[ C_n^k = C_{n-1}^k + C_{n-1}^{k-1}.\]

\newpar{C.1-8}
Note that at the edges of the triangle $k = 1$ or $k = n-1$ thus it's
always equal to $1$.
\begin{verbatim}
            1
          1   1
        1   2   1
      1   3   3   1
    1   4   6   4   1
  1   5   10  10  5   1
1   6   15  20  15  6   1  
\end{verbatim}

\newpar{C.1-9}
We have
\begin{eqnarray*}
  \sum_{i=1}^n i &=& \frac{n(n+1)}{2} \\
  &=& \frac{(n+1)!}{2!(n-1)!} \\
  &=& C_{n+1}^2
\end{eqnarray*}

\newpar{C.1-10}
For $0 \le k \le n$, given that $C_n^k = C_n^{n-k}$, we could limit
our case to $0 \le k \le n/2$.  If $ k < n/2$, we have,
\[ \frac{k!(n-k)!}{(k+1)!(n-k-1)!} = \frac{n-k}{k+1} \ge
\frac{k+1}{k+1} = 1.\]
Thus,
\[ C_n^k \le C_n^{k+1}.\]
We then conclude easily that maximum value is achieved when $k =
\lfloor n/2\rfloor$ when $k$ is even and $k = \lceil n/2\rceil$ if $k$
is odd.

\newpar{C.1-11}
Let $n \ge 0, j \ge 0, k \ge 0$ and $j+k \le n$.  We have,
\begin{eqnarray*}
  C_n^{j+k} &=& \frac{n!}{(j+k)!(n-j-k)!} \\
  &=& \frac{n!}{j!(n-j)!} \times
  \frac{(n-j)!}{k!(n-j-k)!} \times
  \frac{j!k!}{(j+k)!} \\
  &=& \frac{C_n^j C_{n-j}^k}{C_{j+k}^j} \\
  &\le& C_n^j C_{n-j}^k
\end{eqnarray*}

Suppose we choose $j$ items from $n$ items and $k$ items from the $n-j$
remaining items.  Note that here, the elements of the first set could
not be permuted with the elements of the second one without changing
the partition.  Whereas it's possible to permute the elements if we
have chosen directly $j+k$ items.  Thus, we then deduce that
\[ C_n^{j+k} \le C_n^j C_{n-j}^k.\]

In particular, if $k=j=1$, and $n > 1$ we have
\[C_n^2 = \frac{n(n-1)}{2} < C_1^n C_1^{n-1} = n(n-1).\]

\newpar{C.1-12}
Let's show by induction on $k \le n/2$ that
\[ C_n^k \le \frac{n^n}{k^k(n-k)^{n-k}}.\]
For $k = 0$ we have,
\[ C_n^0 = 1 = \frac{n^n}{0^0n^n}.\]

Suppose we have the inequality for $k \le n/2 - 1$.  Then we have
\begin{eqnarray*}
  C_n^{k+1} &=& \frac{n!}{(k+1)!(n-k-1)!} \\
  &=& \frac{n-k}{k+1}C_n^k \\
  &\le& \frac{n-k}{k+1} \times
  \frac{n^n}{k^k(n-k)^{n-k}}, \mbox{by induction} \\
  &=& \frac{n^n}{(k+1)^{k+1}(n-k-1)^{n-k-1}} \times
  \frac{\left(1+\frac{1}{k}\right)^k}
       {\left(1+\frac{1}{n-k-1}\right)^{n-k-1}}
\end{eqnarray*}
Let's consider the function $f: x \mapsto \frac{\ln(1+x)}{x}$ for $x
\ge 0$.  We have,
\[ f'(x) = \frac{\frac{x}{1+x} - \ln(1+x)}{x^2}.\]
If we note $g(x) = x^2 f'(x)$, we have
\begin{eqnarray*}
  g'(x) &=& \frac{1}{(1+x)^2} - \frac{1}{1+x} \ge 0,\ \mbox{and} \\
  g'(0) &=& 0.
\end{eqnarray*}
Thus $g$ is a positive function and so is $f'$.  We then deduce that
$f$ is an increasing function.

Given that $k < n/2 - 1$, we have $k < n-k-1$ thus $f(k) < f(n-k-1)$.
So finally,
\[ C_n^{k+1} \le \frac{n^n}{(k+1)^{k+1}(n-k-1)^{n-k-1}}.\]

Moreover if $k > n/2$, we have $n-k < n/2$.  And from C.3, we deduce
that we have the inequality too.

\newpar{C.1-13}
Using Striling's approximation we have,
\begin{eqnarray*}
  C_{2n}^n &=& \frac{(2n)!}{(n!)^2} \\
  &=& \frac{(2n/e)^{2n}\sqrt{4\pi n}(1+O(1/2n))}
  {(n/e)^{2n} 2\pi n(1+O(1/n))^2} \\  
  &=& \frac{2^{2n}}{\sqrt{\pi n}}(1 + O(1/n))
\end{eqnarray*}

\newpar{C-1-14}
Since $H(1-\lambda) = H(\lambda)$, we could assume that $0 \le \lambda
\le 1/2$.  We then have $\lambda \le 1 - \lambda$, thus
\[  H'(\lambda)  = - \lg \lambda - \frac{1}{\ln 2} +
\lg(1-\lambda) + \frac{1}{\ln 2} \ge 0.\]
Thus, $H(\lambda) \le H(\frac{1}{2}) = 1$.

\newpar{C.1-15}
For $n\ge 0$, we have
\begin{eqnarray*}
  \sum_{k=0}^n C_n^k k &=& \sum_{k=1}^n C_n^k k \\
  &=& \sum_{k=1}^n k \frac{n!}{k!(n-k)!} \\
  &=& n \sum_{k=1}^n C_{n-1}^{k-1} \\
  &=& n \sum_{k=0}^n-1C_{n-1}^k \\
  &=& n 2^{n-1}
\end{eqnarray*}
\end{document}
