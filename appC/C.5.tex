\documentclass[a4paper,12pt]{article}
\usepackage{algorithmic}
\newcommand{\newpar}[1]
{\bigskip \noindent \textbf{Exercises #1} \newline}
\newcommand{\newprob}[1]
{\bigskip \noindent \textbf{Problem #1} \newline}
\newcommand{\subpar}[1]
{\medskip \noindent #1.}
\newcommand{\la}{\leftarrow}
\newcommand{\ra}{\rightarrow}
\newcommand{\prob}[1]{\mathrm{Pr}\left\{ #1 \right\}}
  
\begin{document}
\newpar{C.5-1}
The probability to obtain no heads when you flip a fair coin $n$ times
is $\frac{1}{2^n}$.  If we note $X$ the random variable representing
the total number of successes, the probability to obtain fewer than
$n$ heads when you flip the coin $4n$ times is:
\begin{eqnarray*}
  \prob{X \le n} &\le& \frac{C_{4n}^{2n}}{2^{4n}},\,\mbox{from
    \emph{theorem C.2}} \\ &\le&
  \frac{2^{nH(1/2)}}{2^{4n}},\,\mbox{from (C.7)} \\ &=&
  \frac{1}{2^n}
\end{eqnarray*}

\newpar{C.5-2}
Suppose $np < k < n$, thus $0 < n-k < nq$.  Using \emph{theorem C.4}
with a probability of success $q$ and $Y$ represents the total number
of successes for this trial, we have
\begin{eqnarray*}
  \prob{X > k} &=& \prob{Y < n-k} \\
  &=& \sum_{i=0}^{n-k-1}b(i; n, q) \\
  &<& \frac{(n-k)p}{nq - (n-k)} \,b(n-k;n,q) \\
  &=& \frac{(n-k)p}{k - np}\,b(k;n,p)
\end{eqnarray*}

Suppose: $(np+n)/2 < k < n$, thus $0 < n - k < nq/2$.  From
\emph{corollary C.5}, we deduce the probability
of fewer than $n-k$ failures is less than one half of the probability
of fewer than $n-k+1$ failures.  So the probability of more than $k$
successes is less than one half of the probability of more than $k-1$
successes.

\newpar{C.5-3}
Suppose $a>0$ and $0 < k < n$.  We have for $1\le i\le k-1$
\begin{eqnarray*}
  C_n^i a^i &=& (1+a)^n C_n^i \left(\frac{a}{1+a}\right)^i
  \left(\frac{1}{1+a}\right)^{n-i} \\
  &=& (1+a)^n C_n^i \left(\frac{a}{1+a}\right)^i
  \left(1 - \frac{1}{1+a}\right)^{n-i} \\
  &=& (1+a)^n b(i; n, a/(a+1))
\end{eqnarray*}
For $0 < k < \frac{a}{1+a}n$, using \emph{theorem C.4}, we have
\begin{eqnarray*}
  \sum_{i=0}^{k-1}C_n^i a^i &<&
  (1+a)^n \frac{k/(a+1)}{na/(a+1) - k} b(k; n, a/(a+1) \\
  &=& (1+a)^n \frac{k}{na - k(a+1)}b(k; n, a/(a+1))
\end{eqnarray*}

\newpar{C.5-4}
Using \emph{theorem C.4}, we have
\begin{eqnarray*}
  \sum_{i=0}^{k-1}p^iq^{n-i} &\le& \sum_{i=0}^{k-1}b(i;n,p) \\
  &<& \frac{kq}{np-k}C_n^k p^k q^{n-k} \\
  &\le& \frac{kq}{np-k} \left(\frac{np}{k}\right)^k
  \left(\frac{nq}{n-k}\right)^{n-k},\,\mbox{from (C.6)}
\end{eqnarray*}

\newpar{C.5-5}
Note $X'$ the random variable describing the total number of
failures.  We have $X' = n - X$ and $E[X'] - n - \mu$.  Thus
\begin{eqnarray*}
  \prob{\mu - X \ge r} &=& \prob{X' - (n -\mu) \ge r} \\
  &\le& \left(\frac{(n - \mu)e}{r}\right)^r
\end{eqnarray*}
The same reasoning apply to \emph{Corollary C.9}.

\newpar{C.5-6}
For $0 \le p \le 1$ and $q = 1-p$, let's show that
\[ p e^{\alpha q} + q e^{-\alpha p} \le e^{\alpha^2/2}.\]
We have
\begin{eqnarray*}
  p e^{\alpha q} + q e^{-\alpha p} &=&
  e^{\alpha q}(1 + q(e^{-\alpha} - 1)) \\ &\le& 
  e^{\alpha q}\exp\left(q(e^{-\alpha}-1)\right) \\ &=&
  \exp\left(q(e^{-\alpha}-1+\alpha)\right) \\ &\le&
  \exp\left(q \alpha^2/2\right) \\ &\le&
  e^{\alpha^2/2}
\end{eqnarray*}
By using this inequality instead of (C.43), we obtain
\[ E[e^{\alpha(X-\mu)}] \le \exp(n\alpha^2/2).\]
And finally,
\begin{eqnarray*}
  \prob{X - \mu \ge r} &\le& \exp\left(n\frac{\alpha^2}{2} - \alpha
  r\right) \\ &=&
  \exp\left(\frac{n}{2}\left(\alpha^2 - \frac{2\alpha
    r}{n}\right)\right) \\ &=&
  \exp\left(\frac{n}{2}\left(\left(\alpha - \frac{r}{n}\right)^2
  - \frac{r^2}{n^2}\right)\right) \\
  &=& e^{-r^2/2n},\,\mbox{for $\alpha = r/n$}
\end{eqnarray*}

\newpar{C.5-7}
Note $f(\alpha) = \mu e^\alpha - \alpha r$.
We have $f'(\alpha) \ge 0$ if and only if $\alpha \ge \ln(r/\mu)$.
Thus $f(\ln(r/\mu))$ is the minimum of $f$.

\newprob{C-1: balls and Bins}
\subpar{a}  For each ball, we have $b$ choices of bins, so the number
of ways of placing the balls in the bin is $b^n$.

\subpar{b}  Suppose that we arrange $n$ distinct balls and $b-1$
indistinguishable sticks in a row where the beginning of the row and the
first stick, two consecutive sticks, and the last stick and the end of
the row represent a bin.  Thus we have $(b+n-1)!/(b-1)!$ to place the
distinguishable ordered balls in the bins.

\subpar{c}  If the balls are identical, we need to divide the result
in \textbf{b} by the number of permutations of the balls.

\subpar{d}  To be able to place the balls in the bins, we need that
there are more bins than balls.  And the number of ways to do that is
to choose $n$ bins where to place the balls, because the balls are
identical.

\subpar{e}  Let's use the same reasoning as in \textbf{b}.  Since the
leftmost bin isn't empty, the first element in the row should be a
ball.  That left us with $n-1$ balls and $b-1$ sticks.  Since we
couldn't have two consecutive sticks, each stick should be followed by
a ball.  So we have finally $b-1$ pairs of a stick and a ball and
$n-b$ balls.  The number of ways is thus
\[ \frac{(n-b+b-1)!}{(n-b)!(b-1)!} = C_{n-1}^{b-1}.\]
\end{document}
