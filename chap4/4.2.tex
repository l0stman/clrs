\documentclass[a4paper,12pt]{article}
\usepackage{algorithmic}
\newcommand{\newpar}[1]
{\bigskip \noindent \textbf{Exercises #1} \newline}
\newcommand{\newprob}[1]
{\bigskip \noindent \textbf{Problem #1} \newline}
\newcommand{\subpar}[1]
{\medskip \noindent #1.}
\newcommand{\la}{\leftarrow}
\newcommand{\ra}{\rightarrow}

\begin{document}
\newpar{4-2.1}
Let's show that $T(n) = O(n^{\lg 3})$, that is we have
\[ T(n) \le c\,n^{\lg 3} - n/2\]
for an appropriate choice of the positive constants $c$ .
Suppose we have the inequality for $\lfloor n/2\rfloor$.  Thus,
\begin{eqnarray*}
T(n) &=& 3T(\lfloor n/2\rfloor) + n \\
&\le& 3c(\lfloor n/2\rfloor)^{\lg 3} - 3/2\,n+ n \\
&\le& 3c(n/2)^{\lg 3} - n/2  \\
&=& c\,n^{\lg 3} - n/2
\end{eqnarray*}
And we need to choose the constant $c$ large enough such that,
\[ T(1) \le c - 1/2.\]

\newpar{4.2-2}
Consider the recursion tree of $T(n) = T(n/3) + T(2n/3) + cn$.  The
shortest path from the root to a leaf is the leftmost one which has a
height equal to $\log_3 n$.  So if we consider the subtree of height
$\log_3 n-1$, it's a binary tree so each level has a total cost equal
to $c\,n$.  We then deduce that,
\[ T(n) \ge c\,n (\log_3 n - 1).\]
Thus $T(n) = \Omega(n\lg n)$.

\newpar{4.2-3} We suppose that $T(n) = \Theta(n^2).$ Let's show that
$T(n) \le Cn^2 - b\,n$ for an appropriate choice of the positive
constants $C$ and $b$.  Suppose that we have the inequality for $\lfloor
n/2\rfloor$, thus
\begin{eqnarray*}
T(n) &=& 4 T(\lfloor n/2\rfloor) + c\,n \\
&\le& 4C(\lfloor n/2\rfloor)^2 - 4b\lfloor n/2\rfloor + c\,n \\
&\le& C n^2 - 4b(n/2 - 1) + c\,n \\
&=& C n^2 - b\,n - ((b-c)n - 4b)
\end{eqnarray*}
If we have $b > c$, and for $n$ sufficiently large, that is $n \ge
4b/(b-c)$, we have
\[ T(n) \le n^2 - b\,n.\]
Plus if we note $n_0 = \lfloor 4b/(b-c)\rfloor$, we choose $C$ large
enough so we have,
\[ T(n_0) \le Cn_0^2 - b\,n_0.\]
Thus, we have $T(n) = O(n^2)$.  And let's show that $Dn^2 \le T(n)$
for an appropriate choice of the constant $D > 0$.  Suppose we have
the inequality for $\lfloor n/2\rfloor$, thus
\begin{eqnarray*}
T(n) &=& 4 T(\lfloor n/2\rfloor) + c\,n \\
&\ge& 4 D(\lfloor n/2\rfloor)^2 + c\,n \\
&\ge& 4 D(n/2 - 1)^2 + c\,n \\
&\ge& Dn^2 - 4Dn + c\,n \\
&=& Dn^2 + (c - 4D) n \\
&\ge& Dn^2
\end{eqnarray*}
if we have $D \le c/4$.  Plus, we should choose $D$ small enough such
that,
\[ T(1) \ge D.\]
We then deduce that $T(n) = \Omega(n^2)$.  So finally $T(n) =
\Theta(n^2)$.

\newpar{4.2-4} We want to show that $T(n) = \Theta(n^2)$.  Let's show
that $T(n) \le Cn^2$ for an appropriate choice of the positive
constant $C$.  Suppose that we have the inequality for $n-a$,
thus
\begin{eqnarray*}
T(n) &=& T(n-a) + T(a) + c\,n \\
&\le& C(n-a)^2 + T(a) + c\,n \\
&=& Cn^2 - ((2aC - c)n - Ca^2 - T(a)) \\
&\le& Cn^2
\end{eqnarray*}
if we have 
\[C > \frac{c}{2a}\ \mbox{and}\ \
n \ge n_0 = \left\lceil
\frac{Ca^2+T(a)}{2aC - c}\right\rceil.\]
And we should choose $C$ large enough such that
\[ T(n_0) \le Cn_0^2.\]
We then deduce that $T(n) = O(n^2)$.

\medskip
Let's show that $Dn^2 \le T(n)$ for an appropriate choice of the
positive constant $D$.  Suppose we have the inequality for $n-a$, thus
\begin{eqnarray*}
T(n) &=& T(n-a) + T(a) + c\,n \\
&\ge& C(n-a)^2 + T(a) + c\,n \\
&\ge& Cn^2 + (c-2aC)n \\
&\ge& Cn^2
\end{eqnarray*}
if we have $C \le \frac{c}{2a}$.  Plus $C$ should be small enough such
that
\[ T(a) \ge Ca^2.\]
We then deduce that $T(n) = \Omega(n^2)$.  So finally, we have $T(n) =
\Theta(n^2)$.

\newpar{4.2-5}
Since we can replace $a$ with $1-a$ without changing the recurrence 
\[ T(n) = T(an) + T((1-a)n) + cn\]
we can assume that $0 < a \le 1/2$.

The shortest path in the recursion tree is the leftmost one, which has a
height equal to $-\log_an$.  Thus, if we consider the subtree of
height $-\log_an - 1$, we have a binary and the total cost of each level
is equal to $c\,n$.  We then deduce that,
\[ T(n) \ge c\,n (- \log_a n - 1).\]
So, $T(n) = \Omega(n\lg n)$.

\medskip
Let's show that $T(n) \le C n\lg n$ for an appropriate choice of the
positive constant $C$.  Suppose we have the inequality for $an$ and
$(1-a)n$, we then deduce
\begin{eqnarray*}
T(n) &=& T(an) + T((1-a)n) + cn \\
&\le& Ca\,n\lg(an) + C(1-a)n\lg((1-a)n) + cn \\
&=& Cn\lg n - (C(-a \lg a -(1-a)\lg(1-a)) - c) n \\
&\le& Cn\lg n
\end{eqnarray*}
if we have
\[ C \ge \frac{c}{-a\lg a - (1-a)\lg(1-a)}.\]
Plus, we should choose $C$ large enough such that,
\[ T(2) \le 2C.\]
Thus, $T(n) = O(n\lg n)$.  So finally, $T(n) = \Theta(n\lg n)$.
\end{document}
