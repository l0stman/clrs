\documentclass[a4paper,12pt]{article}
\usepackage{algorithmic}
\newcommand{\newpar}[1]
{\bigskip \noindent \textbf{Exercises #1} \newline}
\newcommand{\newprob}[1]
{\bigskip \noindent \textbf{Problem #1} \newline}
\newcommand{\subpar}[1]{\medskip \noindent #1.}
\newcommand{\la}{\leftarrow}
\newcommand{\ra}{\rightarrow}
\newcommand{\exchange}[2]{\mathrm{exchange}\ #1 \leftrightarrow #2}

\begin{document}

\newpar{6.4-2}
Let's show the property is a loop invariant.
\begin{itemize}
\item \textbf{Initialization: }  For $i = \mathrm{length}[A]$,
  $A[1\ldots i]$ is max-heap containing the $i$ smallest elements of
  $A[1\ldots n]$. And $A[i+1\ldots n]$ is an empty array so we can say
  whatever we want about it.

\item \textbf{Maintenace: } Since $A$ is a max-heap, $A[1]$ is the
  greatest elements of the heap $A$.  Since $A[i+1\ldots n]$ is sorted
  and contains the $n-i$ largest elements of $A[1\ldots n]$, after we
  execute line $3$, the subarray $A[i\ldots n]$ is sorted and contains
  the $n-i+1$ largest elements of $A[1\ldots n]$.  Line $5$ ensures
  that $A$ is a max-heap since the left node and right node are.
  Thus, we maintain the property if we decrease the value of $i$.

\item \textbf{Termination: } At the end of the loop $i = 1$, thus
  $A[1]$ is the smallest elements of $A[1\ldots n]$ and the subarray
  $A[2\ldots n]$ is sorted.  Thus the array $A$ is sorted.
\end{itemize}

\newpar{6.4-3}
See \textbf{Exercises 6.4-5}.

\newpar{6.4-4}
See \textbf{Exercises 6.4-5}.

\newpar{6.4-5} Note $h(1), h(2), \ldots, h(n)$ the distinct elements
of a max-heap of size $n$.  For $k$ between $1$ and $n$, note
$r(h(k))$ the rank of $h(k)$ if the heap is sorted in ascending order.
When the loop variable $i$ is equal to $r(h(k))$, the root of the
max-heap will be equal to $h(k)$ after we execute line $5$ because
it's the greatest element in the max-heap of size $r(h(k))$.

Thus we moved $h(k)$ from position $k$ to position $1$.  This needs at
least $\lfloor \lg k\rfloor$ operations.  Thus to sort the array we
need at least $\sum_{k=1}^n \lfloor \lg k\rfloor$ operations.

We have
\begin{eqnarray*}
  \sum_{k=1}^n \lfloor \lg k\rfloor &>&
  \sum_{k=1}^n \lg k - n \\
  &\ge& \sum_{k=2}^n \int_{k-1}^k \lg x\,dx - n \\
  &=& \int_1^n \lg x\,dx - n \\
  &=& n \lg n - n
\end{eqnarray*}
We then deduce that the running time of \textsc{heapsort} on an array
of distinct elements is $\Omega(n\lg n)$.

\end{document}
