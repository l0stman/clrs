\documentclass[a4paper,12pt]{article}
\usepackage{algorithmic}
\newcommand{\newpar}[1]
{\bigskip \noindent \textbf{Exercises #1} \newline}
\newcommand{\newprob}[1]
{\bigskip \noindent \textbf{Problem #1} \newline}
\newcommand{\subpar}[1]{\medskip \noindent #1.}
\newcommand{\la}{\leftarrow}
\newcommand{\ra}{\rightarrow}
\newcommand{\prob}[1]{\mathrm{Pr}\left\{ #1 \right\}}

\begin{document}

\newpar{6.1-1}
The maximum numbers of elements in a heap of heigh $h$ is
\[ 1 + 2 + 2^2 + \cdots + 2^h = 2^{h+1} - 1.\]
And the minimum numbers of elements is
\[ 1 + 2 + 2^2 + \cdots + 2^{h-1} + 1 = 2^h.\]

\newpar{6.1-2}
If we note $h$ the height of the heap, from \textbf{Exercises 6.1-1},
we have
\[ 2^h \le n < 2^{h+1} - 1\]
Thus
\[ h \le \lg n < h+1.\]
So finally $h = \lfloor \lg n\rfloor.$

\newpar{6.1-3}
Let's show the property by induction in height $h$ of the max-heap.

The property is true for $h=0$.  Suppose it's true of all max-heap of
height $n$.  And let's consider a max-heap of height $n+1$.

By induction, the root $L$ of the left subtree of the root is the
largest value in the left subtree.  We have the same property for the
root $R$ of the right subtree. Since we have a max-heap, $L$ and $R$
are less than their parent which is the root the heap.

\newpar{6.1-4}
If all the elements of a max-heap are distinct, the smallest element
resides on the lowest level.

\newpar{6.1-5}
Yes.

\newpar{6.1-6}
No because $A[\textsc{parent}(9)] = 6 < A[9] = 7.$

\newpar{6.1-7}
The first leaf of the tree is the node following the parent of the
$n^{th}$ node, which is $\lfloor n/2\rfloor + 1$.

\end{document}
