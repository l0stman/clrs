\documentclass[a4paper,12pt]{article}
\newcommand{\newpar}[1]{\bigskip \noindent \textbf{Exercises #1} \newline}
\newcommand{\la}{\leftarrow}
\newcommand{\ra}{\rightarrow}
\begin{document}

\newpar{1.1-1}
Sorting is used for a website search engine wich returns its results
by alphabetical order or by date.

\medskip
Most of the time, in modelisation and numerical simulation of linear
system such as a heat transfert trough differnet layers of a wall,
multiplying matrices is involved.

\medskip
Convex hull is used in pattern recognition and image processing.

\newpar{1.1-2}
We could also measure memory usage.

\newpar{1.1-3} These are some data structures.
\begin{description}
\item[List:]
The advantage of list is it's quite flexible to contain heterogeneous
datas and if we don't know that much about the data we're
manipulating.  Adding and removing a new element by pushing and
popping is efficient.  Changing value on the list is costly.

\item[Vector:]
A vector is preferred over a list when we know in advance the number
of elements we want to manipulate.  Access-time in a vector is
constant.

\item[Hash table:]
A hash table is used to store effeciently a pair of keys and values.
Access time for a hash table is generally faster than for a list,
except for some small datas.  The tricky part is most of the time to
choose a good hash function.
\end{description}

\newpar{1.1-4}
Both of them try to minimize the path length.  But for the
shortest-path you just need the most efficient path between two
points.  Whereas, for the traveling-salesman problem,  the salesman
must pass by a given set of points and go back at the initial point.

\newpar{1.1-5}
In NP-hard problems such as the traveling-salesman,  an
``approximately'' the best solution is good enough.  But in critical
situation such an auto-pilot for an aircraft, only the best solution
will do.
\end{document}
