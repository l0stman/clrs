\documentclass[a4paper,12pt]{article}
\usepackage{algorithmic}
\newcommand{\newpar}[1]
{\bigskip \noindent \textbf{Exercises #1} \newline}
\newcommand{\newprob}[1]
{\bigskip \noindent \textbf{Problem #1} \newline}
\newcommand{\subpar}[1]{\medskip \noindent #1.}
\newcommand{\la}{\leftarrow}
\newcommand{\ra}{\rightarrow}
\newcommand{\exchange}[2]{\mathrm{exchange}\ #1 \leftrightarrow #2}
\newenvironment{alg}[2]
               {\noindent $\textsc{#1}(#2)$ \begin{algorithmic}}
               {\end{algorithmic}}

\begin{document}

\newpar{8.1-1} The smallest depth of a leaf in a decision tree for a
comparison sort is $n-1$.  It's the case for example when the array is
already sorted and we compare successively the $i^{th}$ element with
the $(i+1)^{th}$ element when $i$ varies between $1$ and $n-1$.  We
can show this by induction.

\newpar{8.1-2} We have

\begin{eqnarray*}
  \sum_{k=1}^n \int_{k-1}^k \lg x\, dx &\le & \sum_{k=1}^n \lg k \le
  \sum_{k=1}^n \int_k^{k+1} \lg x\,dx \\
  \int_0^n \lg x\,dx &\le& \sum_{k=1}^n \lg k \le \int_1^{n+1}\lg
  x\, dx \\
  n\lg n - n &\le& \sum_{k=1}^n \lg k \le (n+1)\lg(n+1) - n
\end{eqnarray*}

\newpar{8.1-3}  Suppose that there is a comparison sort whose running
time is linear for at least half of the $n!$ inputs of length $n$.
Thus, at least half of the $n!$ inputs are the leaves of a binary tree
of height $O(n)$.  There exists $c > 0$ such that

\begin{eqnarray*}
  \frac{n!}{2} &\le& 2^{c\,n} \\
  \lg n! &\le& c\,n + 1.
\end{eqnarray*}

Which is absurd since $\lg n! = \Omega(n\lg n)$.

\medskip
The same reasoning applies for a fraction of $1/n$ or $1/2^n$ of the
inputs of length $n$.

\newpar{8.1-4} Each subsequence has $k$! possible permutations.  Thus
the number of reachable leaves in the decision tree of a sorting
algorithm is at least
\[ k! \times k! \times \cdots \times k! = {k!}^{\frac{n}{k}} .\]

If we note $h$ the height of the decision tree, we know that the
latter has no more than $2^h$ leaves.  Hence,
\[ {k!}^{\frac{n}{k}} \le 2^h.\]

Thus,
\begin{eqnarray*}
  h &\ge& \frac{n}{k} \lg k! \\
  &\ge& \frac{n}{k} \times \Omega(k \lg k) \\
  &=& \Omega(n \lg k)
\end{eqnarray*}
And since a lower bound on the height of the decision tree is a lower
bound on the worst-case running time of the algorithm, we deduce that
the latter is $\Omega(n \lg k)$.
\end{document}

